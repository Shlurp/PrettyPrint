\documentclass[10pt]{article}

\usepackage[margin=1.5cm]{geometry}
\usepackage{listings}

\input prettyprint

\initpps

\def\PP{{\scshape PrettyPrint}}

\def\ppsection#1{\begin{blankpp} {\bfseries\Large #1} \end{blankpp}}

\font\bftt=rm-lmtk10

\definecolor{showcasecolor}{RGB}{0,100,255}
\def\showcase{\medskip\bgroup \color{showcasecolor} \leavevmode\vrule width 1pt \strut \aftergroup\medskip \catcode`)=2 \catcode`\{=12 \catcode`\}=12 \catcode`\\=12 \bftt\ \let\next=}
\def\macrocalls{\bgroup \color{showcasecolor} \catcode`)=2 \catcode`\{=12 \catcode`\}=12 \catcode`\\=12 \tt \let\next=}

% Kinda just took this from Overleaf

\definecolor{codegreen}{rgb}{0,0.6,0}
\definecolor{codegray}{rgb}{0.5,0.5,0.5}
\definecolor{codepurple}{rgb}{0.58,0,0.82}
\definecolor{backcolour}{rgb}{0.95,0.95,0.92}

\lstdefinestyle{mystyle}{
    backgroundcolor=\color{backcolour},   
    commentstyle=\color{codegreen},
    keywordstyle=\color{magenta},
    numberstyle=\tiny\color{codegray},
    stringstyle=\color{codepurple},
    basicstyle=\ttfamily\footnotesize,
    breakatwhitespace=false,         
    breaklines=true,                 
    captionpos=b,                    
    keepspaces=true,                 
    numbers=left,                    
    numbersep=5pt,                  
    showspaces=false,                
    showstringspaces=false,
    showtabs=false,                  
    tabsize=2
}

\lstset{style=mystyle}

\begin{document}

\barcolorbox{255,200,200}{150,0,0}{255,150,150}{
	\hfil{\Huge\scshape PrettyPrint}\hfil

	\hfil{\scshape Macros and Environments for Printing Pretty}\hfil

	\medskip
	\hfil{\small Slurp}\hfil
}

\bigskip

The PrettyPrint package provides several macros for creating pretty boxes of text.
I do not recommend you use it, and rather use a better and more established package like tcolorbox instead.

\ppsection{Installation}

Download this git repository, and move the {\tt prettyprint.tex} file either into your working directory (where your \TeX\ 
project is) or into your {\tt texmf} tree (something like {\tt\catcode`\~=12 ~/texmf/tex/local}).
Then in your \TeX\ file add the line

\begin{lstlisting}[language=tex]
	\input prettyprint
\end{lstlisting}

\ppsection{Environments}

\showcase(\begin{ppbox}{<bg color>}{<fg color>}{<bar color>} material... \end{ppbox})

This creates a colored box with a line on the left side of color {\tt <bar color>}, and puts {\tt material...} in it.
The box spans the width of the document and can break over pages.
So for example the following yields:

\begin{lstlisting}[language=tex]
	\begin{ppbox}{blue!20!white}{blue!70!white}{blue!50!white}
		Lorem ipsum dolor sit amet, consectetur adipiscing elit, sed do eiusmod tempor incididunt ut labore et dolore magna aliqua.
		Ut enim ad minim veniam, quis nostrud exercitation ullamco laboris nisi ut aliquip ex ea commodo consequat.
		Duis aute irure dolor in reprehenderit in voluptate velit esse cillum dolore eu fugiat nulla pariatur.
		Excepteur sint occaecat cupidatat non proident, sunt in culpa qui officia deserunt mollit anim id est laborum.
	\end{ppbox}
\end{lstlisting}

\begin{ppbox}{blue!20!white}{blue!70!white}{blue!50!white}
	Lorem ipsum dolor sit amet, consectetur adipiscing elit, sed do eiusmod tempor incididunt ut labore et dolore magna aliqua.
	Ut enim ad minim veniam, quis nostrud exercitation ullamco laboris nisi ut aliquip ex ea commodo consequat.
	Duis aute irure dolor in reprehenderit in voluptate velit esse cillum dolore eu fugiat nulla pariatur.
	Excepteur sint occaecat cupidatat non proident, sunt in culpa qui officia deserunt mollit anim id est laborum.
\end{ppbox}

\ppsection{Macros}

\showcase(\barcolorbox{<bg color>}{<fg color>}{<bar color>}{<text>...})

This creates a colored box with a rectangular border (like the one at the beginning of this document).
The colors must all be in an RGB style.
The box spans the width of the document, not the width of the text inside.
The box cannot break.
So for example, the following code gives:

\begin{lstlisting}[language=tex]
	\barcolorbox{220,220,255}{0,0,130}{80,80,200}{
		Lorem ipsum dolor sit amet, consectetur adipiscing elit, sed do eiusmod tempor incididunt ut labore et dolore magna aliqua.
		Ut enim ad minim veniam, quis nostrud exercitation ullamco laboris nisi ut aliquip ex ea commodo consequat.
		Duis aute irure dolor in reprehenderit in voluptate velit esse cillum dolore eu fugiat nulla pariatur.
		Excepteur sint occaecat cupidatat non proident, sunt in culpa qui officia deserunt mollit anim id est laborum.
	}
\end{lstlisting}

\barcolorbox{220,220,255}{0,0,130}{80,80,200}{
	Lorem ipsum dolor sit amet, consectetur adipiscing elit, sed do eiusmod tempor incididunt ut labore et dolore magna aliqua.
	Ut enim ad minim veniam, quis nostrud exercitation ullamco laboris nisi ut aliquip ex ea commodo consequat.
	Duis aute irure dolor in reprehenderit in voluptate velit esse cillum dolore eu fugiat nulla pariatur.
	Excepteur sint occaecat cupidatat non proident, sunt in culpa qui officia deserunt mollit anim id est laborum.
}

In the future this will become an environment, but I'm too lazy at the moment.

\showcase(\newpp{<name>}{<bg color>}{<fg color>}{<bar color>})

Creates an environment by the name {\tt <name>} which is equivalent to passing {\tt <bg color>}, etc to a {\tt ppbox}.
That is, something like \macrocalls(\newpp{bluebox}{255,200,200}{255,50,50}{255,150,150}) creates an environment called {\tt bluebox}
and calling \macrocalls(\begin{bluebox}) is equivalent to calling {\tt ppbox} with colors defined as those RGB values.

\begin{lstlisting}[language=tex]
	\newpp{bluebox}{255,200,200}{255,50,50}{255,150,150}

	\begin{bluebox}
		Lorem ipsum dolor sit amet, consectetur adipiscing elit, sed do eiusmod tempor incididunt ut labore et dolore magna aliqua.
		Ut enim ad minim veniam, quis nostrud exercitation ullamco laboris nisi ut aliquip ex ea commodo consequat.
		Duis aute irure dolor in reprehenderit in voluptate velit esse cillum dolore eu fugiat nulla pariatur.
		Excepteur sint occaecat cupidatat non proident, sunt in culpa qui officia deserunt mollit anim id est laborum.
	\end{bluebox}
\end{lstlisting}

Yields

\newpp{bluebox}{255,200,200}{255,50,50}{255,150,150}

\begin{bluebox}
	Lorem ipsum dolor sit amet, consectetur adipiscing elit, sed do eiusmod tempor incididunt ut labore et dolore magna aliqua.
	Ut enim ad minim veniam, quis nostrud exercitation ullamco laboris nisi ut aliquip ex ea commodo consequat.
	Duis aute irure dolor in reprehenderit in voluptate velit esse cillum dolore eu fugiat nulla pariatur.
	Excepteur sint occaecat cupidatat non proident, sunt in culpa qui officia deserunt mollit anim id est laborum.
\end{bluebox}

\showcase(\newmathpp{<name>}{<header>}{<bg color>}{<fg color>}{<bar color>})

This is like \macrocalls(\newpp) but instead creates a ``math'' colored box.
At the top of the box the {\tt <header>} will be printed.
It is also possible to pass optional arguments for tags and headers to the {\tt <name>} environment which will allow you to 
reference it in the future and add a header to the box.

\begin{lstlisting}[language=tex]
	\newmathpp{definition}{Definition}{255,200,200}{150,0,0}{255,100,100}

	\begin{definition}[evaldefinition,Eigenvalue\ Definition]
		An eigenvalue of a linear transformation $T$ is a value $\lambda\in K$ such that there exists a non-zero vector $v$
		where
		\[ T(v) = \lambda v \]
	\end{definition}
\end{lstlisting}
\newmathpp{definition}{Definition}{255,200,200}{150,0,0}{255,100,100}
                                                                                                                        
\begin{definition}[evaldefinition,Eigenvalue\ Definition]
	An eigenvalue of a linear transformation $T$ is a value $\lambda\in K$ such that there exists a non-zero vector $v$
	where
	\[ T(v) = \lambda v \]
\end{definition}

\macrocalls(\newmathpp) also creates an environment called {\tt <name>*} which also prints the current math counter, a counter
which increments after every time you call a starred mathpp and resets at every subsection.
So if we use the {\tt definition} mathpp defined above and use:

\begin{lstlisting}[language=tex]
	\begin{definition*}[derivativedefinition,Derivatives]
		If $f$ is a real function defined at a point $x$, then the derivative of $f$ at $x$, denoted $f'(x)$, is defined to be:
		\[ f'(x) = \lim_{h\to0} \frac{f(x+h) - f(x)}{h} \]
		If the limit exists.
	\end{definition*}
\end{lstlisting}

\setcounter{section}{1} \setcounter{subsection}{1}

\begin{definition*}[derivativedefinition,Derivatives]
	If $f$ is a real function defined at a point $x$, then the derivative of $f$ at $x$, denoted $f'(x)$, is defined to be:
	\[ f'(x) = \lim_{h\to0} \frac{f(x+h) - f(x)}{h} \]
	If the limit exists.
\end{definition*}

\showcase(\ppemph{<material>})

This prints {\tt material} in bold and in the same color as the bar of the current pp environment (if it is outside of a pp
environment, it is black).
So for example using the previously defined {\tt definition} environment:

\begin{lstlisting}[language=tex]
	\begin{definition*}
		A set $G$ is called a \ppemph{group} if it is equipped with a binary operation $\circ$ such that $\circ$ is associative,
		has an identity $e\in G$ such that for every $a\in G$ $e\circ a=a\circ e=a$, and for every $a\in G$ there exists an $a^{-1}\in G$
		where $a\circ a^{-1} = a^{-1}\circ a = e$.
	\end{definition*}
\end{lstlisting}

\begin{definition*}
	A set $G$ is called a \ppemph{group} if it is equipped with a binary operation $\circ$ such that $\circ$ is associative,
	has an identity $e\in G$ such that for every $a\in G$ $e\circ a=a\circ e=a$, and for every $a\in G$ there exists an $a^{-1}\in G$
	where $a\circ a^{-1} = a^{-1}\circ a = e$.
\end{definition*}

\showcase(\ppord{<material>})

This is like \macrocalls(\ppemph) but sets the color to the foreground color of the current pp environment (black if not inside a pp
environment).

\showcase(\ppref[<prefix>]{<tag>})

This creates a hyperlink to the mathpp environment tagged with {\tt <tag>}, prepending {\tt prefix} to the hyperlink title if provided.
The hyperlink's title is the mathpp environment's header if it was provided, and otherwise is the pp math counter value of the target.

\begin{lstlisting}[language=tex]
	\begin{definition*}[examplelink]
		This is an example usage of a mathpp which references \ppref[definition]{examplelink} (itself), and \ppref{evaldefinition}.
	\end{definition*}
\end{lstlisting}

\begin{definition*}[examplelink]
	This is an example usage of a mathpp which references \ppref[definition]{examplelink} (itself), and \ppref{evaldefinition}.
\end{definition*}

\showcase(\initpps)

This just defines a bunch of pp environments, literally expands to:

\begin{lstlisting}[language=tex]
	\newmathpp{defn}{Definition}{255,200,200}{150,0,0}{200,100,100}
	\newmathpp{thrm}{Theorem}{200,200,255}{0,0,150}{100,100,200}
	\newmathpp{coro}{Corollary}{220,220,255}{0,0,130}{80,80,200}
	\newmathpp{lemm}{Lemma}{255,200,255}{150,0,150}{200,100,200}
	\newmathpp{prop}{Proposition}{255,240,180}{150,100,0}{200,150,70}
	\newmathpp{exam}{Example}{200,255,200}{0,100,0}{50,150,50}
	\newmathpp{note}{Note}{255,255,200}{150,100,0}{200,150,100}
	\newmathpp{proof}{Proof}{255,255,255}{0,0,0}{0,0,0}
	\newmathpp{claim}{Claim}{255,255,255}{0,0,0}{0,0,0}
	\newmathpp{exercise}{Exercise}{200,255,200}{0,100,0}{50,150,50}
	\newmathpp{solution}{Solution}{255,255,255}{0,0,0}{0,0,0}

	\newpp{blankpp}{255,255,255}{0,0,0}{0,0,0}
\end{lstlisting}

So use it in your preamble or wherever if you'd like.

\end{document}

